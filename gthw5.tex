\documentclass[10pt,a4paper]{article}
\usepackage[utf8]{inputenc}

\title{%
  Game Theory: Homework 4 \\
  \large Silvan Hungerbuehler, 11394013}

\usepackage{mathptmx} % "times new roman"
\usepackage{amssymb}
\usepackage{amsmath, amsthm}
\usepackage{amsfonts}
\usepackage{enumitem}
\usepackage{verbatim}
\usepackage{hyperref}
\usepackage{comment}
\usepackage[margin=1in]{geometry}
\usepackage{float}
\usepackage{tikz}
\usetikzlibrary{positioning}
\usepackage{bm}

\usepackage[normalem]{ulem}
\date{}
\begin{document}
\maketitle

\section*{Question 1}
\subsection*{a}
Every bidder submits a concealed bid to the auctioneer who then awards the item for sale to the highest bidder. This bidder then pays a price equal to the third-highest bid submitted.
\subsection*{b}
Submitting the highest bid as long as the third-highest bid is still less or equal to your true valuation. \\
Interestingly, truth-telling is no longer incentivized as it was in the second-price auction and strategizing becomes now possible. Consider a situation with three bidders: $i,j$ and $h$ where the true valuations are such that $v_j>v_i>v_h$. Say, $j$ and $h$ submit truthful bids, so that $\hat{v_j}=v_j$ and $\hat{v_h}=v_h$. Then $i$ can profit from offering some bid $\hat{v_i}>\hat{v_j}>v_i$. Although $i$ is not playing truthfully, for $\hat{v_i}\neq v_i$, her utility will be $v_i-\hat{v_h}>0$, thus strictly larger than the payoff of zero she would have gotten for playing $\hat{v_i}=v_i$.
\subsection*{c}
\end{document}