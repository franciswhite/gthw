\documentclass[10pt,a4paper]{article}
\usepackage[utf8]{inputenc}

\title{%
  Game Theory: Homework 3 \\
  \large Silvan Hungerbuehler, 11394013}

\usepackage{mathptmx} % "times new roman"
\usepackage{amssymb}
\usepackage{amsmath, amsthm}
\usepackage{amsfonts}
\usepackage{enumitem}
\usepackage{verbatim}
\usepackage{hyperref}
\usepackage{comment}
\usepackage[margin=1in]{geometry}

\usepackage[normalem]{ulem}
\date{}
\begin{document}
\maketitle

\section*{Question 1}


\section*{Question 3}
This Bayesian game can be formally represented as a tuple $<N,\boldsymbol{A}, \Theta, P, u>$. There are two players $N=\{1,2\}$, two actions available to each so that the set of action profiles is $\boldsymbol{A}=\{(T,L),(T,R),(B,L),(B,R)\}$. The players' type encapsulates which of two possible games they are playing. Rowena is certain about which game it is while Colin is not, so $\Theta_1=\{\star\}$,$\Theta_2=\{zero,two\}$. Thus $\Theta=\{(\star,zero),(\star,two)\}$ and $p: \Theta \rightarrow 0.5$. Finally, the utility function can be given most easily in two tables - one for the type profile $(\star,zero)$ and the other for $(\star,two)$.
\begin{table}[h]
\centering
\begin{tabular}[l]{|l|l|l|}
\hline
          & $L$ & $R$  \\ \hline
$T$     & 5,5   & 0,10 \\ \hline
$B$		& 10,0	& 1,1 \\ \hline
\end{tabular}
\quad
\begin{tabular}[r]{|l|l|l|}
\hline
          & $L$ & $R$  \\ \hline
$T$     & 5,5   & 2,10 \\ \hline
$B$		& 10,2	& 1,1 \\ \hline
\end{tabular}
\end{table}

Since Rowena will know which game is played, she can make conditional plans for either case. Her available pure strategies are then $S_1=\{AB,AT,Bi0-Ti2,Ti0,Bi2\}$, or in beautiful prose: ''Play always bottom, Play always top, Play bottom in case $\alpha=0$ and top in case $\alpha=2$, Play bottom in case $\alpha=2$ and top in case $\alpha=0$''. Colin, however, does not have the luxury of conditionalizing his actions on the type profile. His only two pure strategies are $S_2=\{L,R\}$. Given these individual strategy sets, we can represent the pure strategy space by means of a $4\times 2$ matrix. The boxes contain the players' expected payoffs associated with a given pure strategy profile. To provide an example, Colin's expected payoff of $3.5$ in the strategy profile $(Bi2-Ti0,L)$ is the result of him winning $5$ when the $\alpha=0$-game is played - because then Rowena plays $T$, according to her conditional strategy - and him winning $2$ when the $\alpha=2$-game is played - because then Rowena's contingency plan then foresees playing $B$. Each game being equiprobable with probability $0.5$ this yields an expected utility of $0.5*5+0.5*2=3.5$.
\begin{table}[h]
\centering
\begin{tabular}[l]{|l|l|l|}
\hline
          & $L$ & $R$  \\ \hline
$AB$     & 10,1   & 1,1 \\ \hline
$AT$		& 5,5	& 1,10 \\ \hline
$Bi2-Ti0$	& 7.5,3.5 & 0.5, 5.5 \\ \hline
$Bi0-Ti2$	& 7.5,2.5	& 1.5,5.5
\end{tabular}
\end{table}




 
\end{document}