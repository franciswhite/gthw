\documentclass[10pt,a4paper]{article}
\usepackage[utf8]{inputenc}

\title{%
  Game Theory: Homework 1 \\
  \large Due Date: 15.2.17
  \large Silvan Hungerbuehler, 11394013}

\usepackage{mathptmx} % "times new roman"
\usepackage[left=4cm,right=4cm,top=4.5cm,bottom=5.5cm,footskip=1cm]{geometry} % for submission requirements
\usepackage{amssymb}
\usepackage{amsmath, amsthm}
\usepackage{amsfonts}
\usepackage{enumitem}
\usepackage{verbatim}
\usepackage{hyperref}
\usepackage{comment}

\usepackage[normalem]{ulem}
\date{}
\begin{document}
\maketitle

\section*{Question 1}
\subsection*{a)}
There are 2 pure Nash Equilibria (NE) and 1 mixed NE.\\
\textbf{Pure NE}\\
We find the pure NE by analyzing the players' pure strategy profiles. Given a pure strategy profile, if neither player has  an incentive to deviate to another pure strategy then that strategy profile is a NE. A player has an incentive to deviate if she can obtain a strictly higher payoff by playing the other action.\\
This is visualized in diagram below. Horizontal arrow represent the row player's incentive for deviation, vertical arrows the column player's incentive for deviation. The strategy profiles $(T,L)$ and $(B,R)$ are pure NE, for neither player can improve by unilaterally switching her action.

\textbf{Mixed NE}

\subsection*{b)}
Let $p$ be the probability of the row player playing $T$, and $1-p$ consequently the probability of playing $B$. Likewise, $q$ denotes the probability of the column player  playing $L$ and $1-q$ of playing $R$.

Row player chooses $p$ so as to make the column player indifferent between playing $L$ and $R$.\\
\begin{align*}
EU(L)&=EU(R)\\
8\times p + 2\times (1-p)&= 4\times p + 3\times (1-p) \\
6\times p + 2 &= p+3 \\
5p &= 1 \\
p &= \tfrac{1}{5}
\end{align*}




\end{document}